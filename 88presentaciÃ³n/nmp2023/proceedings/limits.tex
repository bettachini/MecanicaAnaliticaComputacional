Traditionally, the physical systems studied in analytical mechanics courses are relatively simple in order to limit the time and/or difficulty of mathematical analysis and/or algebra calculations required by the steps commented in the previous paragraph.
But this extreme simplification leads to a later noticeable jump in the complexity required to model mechanical devices, fluid dynamics, or rigid structures, in the courses, shown in the figure \ref{fig:correlativas}, coming up next where the student must apply this tool.
The limitation to complexity of the examples is imposed by what can be calculated on a blackboard or presentation slides.
The same goes for the length and complexity of the problems that students can be asked to solve.

Computer algebra systems, that will soon turn 60 \cite{ams}, have of course certain limitations on what they can compute, but these are well beyond of the most complex problem an undergraduate student of engineering could meddle in.
Libraries for symbolic algebra and calculus are available for all major general programming languages, both libraries and languages being published as free software.
So the only reason that limits the reach of what students can calculate is a culture that keeps the blackboard, slides or paper as the centre piece of the work at classrooms in place of computers.
Although numerical analysis is reportedly used to solve problem sets \cite{mirasso-raichman, caligaris-rodriguez}, it is rarely used by the teaching staff during classes.
But if the application of numerical analysis during classes is rare, the use of symbolic computer algebra is even rarer.
The use of computer algebra and calculus solutions re-focus the students attention towards the subject matter of the current course.
