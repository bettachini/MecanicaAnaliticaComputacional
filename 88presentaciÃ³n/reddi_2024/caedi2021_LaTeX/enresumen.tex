La exposición de teoría y la ejercitación práctica tomaron distintas ventajas de la metodología de este curso.

Clases de teoría:
\begin{itemize}
    \item La exposición en pizarrón o en una presentación en que los alumnos están concentrados en transcribir lo allí escrito se reemplazó por código que pueden reutilizar para resolver sus ejercicios.
    \item En las clases en línea cada palabra del docente durante la clase queda registrada en video  liberando al alumno de la toma de notas.
    \item El docente puede cambiar el código durante la clase para corregir un error o graficar otro aspecto de la temática.
\end{itemize}

Ejercicios de práctica:
\begin{itemize}
    \item En papel una variación sobre un ejercicio resuelto anteriormente obliga a reiterar tediosos cálculos similares. Con código basta con modificar ligeramente el mismo para atender al nuevo caso.
    \item En forma remota varios alumnos pueden trabajar concurrentemente en la resolución en un mismo ejercicio.
    \item Los alumnos pueden alertar al docente a toda hora vía el LMS de un inconveniente que enfrenten en la resolución de un ejercicio. El docente puede dedicarle tiempo y detenimiento en el momento que encuentre propicio a diferencia del acotado tiempo de consultas del que se dispone en el aula.
    \item Los docentes pueden comentar y corregir el mismo código sobre el que está trabajando el alumno inclusive en tiempo real.
\end{itemize}
