“Fundamentos de programación” y “Cálculo numérico” se mencionan entre los “descriptores de conocimiento” para un Ingeniero Mecánico en la “Propuesta de Estándares de Segunda Generación para la Acreditación de Carreras de Ingeniería en la República Argentina” aprobado por el Consejo Federal de Decanos de Ingeniería (CONFEDI) en 2018, mejor conocido como “Libro Rojo de CONFEDI”. Lamentablemente tras que la teoría y uso de estas herramientas son aprendidos por los alumnos no suelen aprovecharse en profundidad en cursos de años posteriores.

En este trabajo se describe la experiencia que se tuvo en la asignatura Mecánica General del 3.er año de la carrera en la UNLaM. Tradicionalmente los sistemas modelados se limitan a los resolubles analíticamente por trabajar en pizarrón o papel. En este curso, los estudiantes resolvieron sus ejercicios utilizando código en lenguaje Python, haciendo uso de herramientas de este siglo, como bibliotecas de funciones para el cálculo simbólico, numérico, graficación, etc.

Todas las clases se dictaron íntegramente usando cuadernos de Jupyter como plataforma. En estos se intercala código con información gráfica y texto incluyendo una clara notación matemática con simbología LaTeX. Este código es re-utilizable por el estudiante para resolver la ejercitación del curso con la misma herramienta, así como para ser aprovechado en asignaturas futuras y en su vida profesional.

Estos cuadernos se ejecutan sobre software libre. Plataformas web de acceso gratuito a través del navegador permitieron a los estudiantes ejecutarlos en su hogar o trabajo, permitiendo comentar y editar en forma conjunta un mismo cuaderno entre alumnos y/o docentes.

La pandemia nos forzó a enseñar a través de una computadora. Tras un periodo inicial de adaptación los estudiantes reconocieron las virtudes de esta metodología. Inclusive la evaluación fue más enriquecedora que en un curso convencional al alcanzar la complejidad de simular sistemas mecánicos similares a los industriales.\par\bigskip

Abstract\par\medskip

“Numerical Analysis” and “Programming Fundamentals'' are mentioned as “Knowledge Descriptors'' for a Mechanical Engineer in the “Proposal of standards for the second generation for engineering degrees accreditation in the Argentine Republic” (“Propuesta de Estándares de Segunda Generación para la Acreditación de Carreras de Ingeniería en la República Argentina”) approved by the “Federal council of engineering deans” (Consejo Federal de Decanos de Ingeniería, CONFEDI) in 2018, and best known as “Libro Rojo de CONFEDI”. Regrettably after theory and use of these tools are learnt by students they are not fully exploited in courses at later years.

In this work the experience had at the subject Mecánica General (General Mechanics) of the UNLaM’s mechanical engineering degree third year is described. Traditionally modeled systems are limited to those analytically solvable working on blackboard or paper. In this course the students solved their problem sets using Python language code, applying this century tools, such as library functions for symbolic and numerical analysis, plotting, etc.

All classes were conducted in full using Jupyter notebooks as a platform. In those notebooks code  is interbowen with graphical information and text including clear mathematical notation with LaTeX symbols. Students can re-use this same code to solve course’s problem sets as well as in future courses and their professional life.

The notebooks mentioned above run  on free software. Students operate on them with free to use web platforms that  allow concurrent commenting and editing among them and/or their professor.

The pandemic pushed us all to teach through a computer. After an initial adaptation period the students recognized the virtues of this methodology. Even assessments were more enriching than in a conventional course as it reached the complexity of simulating industrial-like mechanical systems.