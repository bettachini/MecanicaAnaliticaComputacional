This paper outlines the methodologies, tools and organisation of mechanical engineering undergraduate course that employs code as the sole means to perform all its calculations. Modeling simple mechanical devices as rigid bodies and employing the analytical mechanics Euler-Lagrange equation the code is able to simulate the dynamics and mechanical efforts of these systems. No prior programming skills are required, provided code that solves example problems is modified by students to address new ones. Jupyter Notebooks running on Google Colaboratory provides an unified platform for lesson's material embedding code among Markdown formatted text and \LaTeX\ mathematical expressions. 

Originally conducted online during the SARS-CoV-2 pandemic, the course has transitioned to in-person sessions and a flipped classroom approach. Exercises sets turn-in is mandatory but keeping overall homework at the minimum to free-up student's time for reading of lessons before weekly face-to-face meetings that provide opportunities for clarification and progress discussions. A learning management system is used to keep track of each student progress and to provide asynchronical consultations.

The course culminates in a challenging final problem: analysing forces and torques on a simplified industrial robotic arm. Beyond technical skills, students enhance presentation and synthesis abilities, defending their solutions through concise oral presentations to teachers.
