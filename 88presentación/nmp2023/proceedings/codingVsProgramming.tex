Having fluency in a computer language's code in order to be able to comprehend what was written by another person, modify it or even write its own is of course an ability required for programming, but by itself fells short of the analytical skills and knowledge of algorithms and computer architecture required by professionals in the field.
For the matters of our course we state a strict distinticion between ``programming'' and the less demanding ``coding'' to the students at the first meeting they have with the teaching staff.
Altough at UNLaM they had a previous course of introduction to programming based on \verb!C++!, so they already posess ability on programming algorithms, that in any way is something they need to apply in the course.
In our experience, even for students that allegedly ``have forgotten'' what they learnt on programming due to lack of use found that Python's concise and simple syntax \cite{perkel_programming_2015} makes it relatively accesible to grasp it to the level required to understand and modify the code provided by the teaching staff.

For each class the teaching staff makes available examples of codes that perform all the procedures required to model a certain characteristic of a mechanical system.
Related exercises are provided in a problem set that can be solved by making small modifications to the code.
This later ability to adapt the code by slight changes is nothing more than code reuse, a common practice in the computing industry since its inception.

A good deal of courses appeared during the last decade that take advantage of tools such as the ones used in our course, to produce graphical output to clarify the results of numerical simulations in diverse areas such as computational fluid dynamics \cite{barba_cfd_2018} or chemical thermodynamics \cite{vallejo_google_2022}.
There are also teaching materials, once again relying on these tools, that are published in a book alike format in the fields of chemistry \cite{weiss_creative_2021}.
But these cited examples used their published material containing code to support the lessons, not to be the basis of their problems sets.
There are of course others that do exactly that, even providing example problems with the code required to solve them embedded into them, such one on stucture modelling for civil engineers \cite{laureline_duvillard_using_nodate}, but these do not rely on the media to be the basis for their lessons.
We understand that not using the same digital format with embedded code both at lessons and to solve the problem sets results in the exercise of transcription discussed previously. 
To the best of our knowledge there is no course, besides ours, that integrates analytical mechanics with numerical analysis in lessons to provide code able to calculate problems of dynamics and efforts in rigid body mechanical systems using the same digital media.
