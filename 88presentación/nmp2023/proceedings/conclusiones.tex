% This is a summary of the main drivers of the course:
This course differs from conventional ones in two ways:
\begin{itemize}
    \item Code-centric% Advantages of Code-Based Learning:
    \begin{itemize}
        \item Avoids the repetitive nature of blackboard or paper based calculations. 
        \item By iteratively modifying previously tested code (initially designed for simpler mechanical systems), students expand their analytical capabilities.
        \item The complexity of the code evolves alongside the mechanical system’s intricacies introduced each class.
        \item This approach eliminates the need to \textit{start from scratch} when dealing with the extensive calculations required for analyzing complex mechanical systems using the Euler-Lagrange formalism.
        \item All systems used are currently available online on a non-cost basis, from the student point of view. Being based on free software, if any of them is later placed behind a paywall, it would be simple to run them from on the premise servers.
    \end{itemize}
    \item Flipped classroom
    \begin{itemize}
        \item Students are provided with online theory and example problems to study before weekly meetings. These asynchronous activities save classroom time for discussions and problem solving.
        \item During synchronic meetings they can rise to teachers any questions related to theory or problem-solving so they can finish their exercise sets.
        \item All exercises are turned-in for evaluation. Compliance is tracked with an online learning management system. 
    \end{itemize}
\end{itemize}
Currently, there is limited statistical data available on the impact of the course and the described methodologies.
However, feedback from students consistently indicates a high level of satisfaction, especially with the code-driven aspect of the course.
Additionally, students express interest in the final examination as it provides an opportunity to apply both their presentation skills and the knowledge acquired throughout the course.

In relation to the flipped classroom model, students acknowledge that it requires more effort, but a majority of them agree that it is a positive and beneficial implementation.
as former students assesed that the tools employed in the course were usuful to them in subsequent subjects and professional lives, the authors had consequently gained confidence on having choosed the current approach over traditional ones.
