The last three quarters, the SARS-CoV-2 pandemic forced the courses for Mechanical Engineering students at the National University of La Matanza (UNLaM) to be taught remotely. The fact that students were behind a computer during class was used to impose a pedagogical methodology that omitted traditional chalkboard, paper, and non-interactive computer presentations in favor of presenting theoretical concepts and practicing their use in activities on an interactive computer platform based on code written in the Python language.

The complexity of the code increases as new aspects affecting a mechanical system are contemplated from class to class. 
In a chalkboard and paper-based course, similar calculations are repeated in each new activity, while in a code-based course, it presents the advantage of its reuse. 
By making modifications to the code tested in previous classes with simple mechanical systems, the analysis capacity is expanded without the loss of time that would be required for writing from scratch the long set of calculations that an analysis in the Euler-Lagrange scheme of a more complex mechanical system requires.


In the post-pandemic return to the classroom the proposed methodology was further improved, now using an inverted classroom aproach \MR{incluir CITA}. The students are presented to online theory and examples to be read and physics problems to be started at home that must be finished during their time in the classroom with the help of teachers. In this way the theory and initial stages of the problem-solving are asynchronous, while they take advantage of the classroom time to ask issues regarding the problems and theory. 



%Source: Conversation with Bing, 11/23/2023
% % (1) How to Translate Languages in Python - Python Code. https://thepythoncode.com/article/translate-text-in-python.
% % (2) Text Translation with Google Translate API in Python - Stack Abuse. https://stackabuse.com/text-translation-with-google-translate-api-in-python/.
% % (3) Translate website to any specific language, on page load. https://stackoverflow.com/questions/13030153/translate-website-to-any-specific-language-on-page-load.
% % (4) undefined. http://domain.example.
% % (5) undefined. http://code.google.com/p/jquery-translate/.

% Los últimos tres cuatrimestres la pandemia de SARS-CoV-2 forzó a que los cursos para los estudiantes de Ingeniería Mecánica en la Universidad Nacional de La Matanza (UNLaM) se dicten en forma remota. El hecho de que los alumnos estén tras una computadora durante la clase se aprovechó para imponer una metodología pedagógica que obvió los tradicionales soportes pizarrón, papel y presentaciones informáticas no interactivas en favor de presentar conceptos teóricos y ejercitar su uso en actividades en una plataforma informática interactiva basada en código escrito en el lenguaje Python.
% La complejidad del código se incrementa a medida que se contemplan clase a clase nuevos aspectos que afectan a un sistema mecánico. En un curso basado en pizarrón y papel se repiten cálculos similares en cada nueva actividad mientras que en un curso basado en código este presenta la ventaja de su reutilización. Realizando modificaciones al código probado en clases anteriores con sistemas mecánicos simples se expande la capacidad de análisis sin la pérdida de tiempo que insumiría una escritura desde cero del largo conjunto de cálculos que insume un análisis en el esquema de Euler-Lagrange de un sistema mecánico más complejo.