The stored-program digital computer was conceived as a more flexible tool for solving numerical analysis problems than their electro-mechanical predecessors that had to be reconfigured to address different problems \cite{kurrer_konrad_2010}.
Originally funded by and reserved for military research programs at the forties, it was towards the end of that decade that civilians at universities started to exploit them for other purposes \cite{simon_h_lavington_history_1975}. 
So, PhD candidates were among the first students to use these mainframe computers in the so called punch card and batch processing era that stretched into the 1970s. 
By then, higher-level languages and lower-cost minicomputers had opened up access for undergraduate students \cite{kemeny_dartmouth_1968}.
Time-sharing based live interaction with computers became an established practise, allowing to serve several students at once to deploy computer-based instruction \cite{tidball_using_1978}.
These computer-assisted instruction systems provided tailor-made lessons created by the teaching staff, alongside questionnaires and other interactive feedback mechanisms to evaluate students' understanding of a variety of subjects. This approach avoided pressuring students into programming \cite{cope_little_2023}.
A contemporary push on the opposite direction directed to school age students sought to foster mathematical and geometrical abilities, creating languages aimed for educational use based on ideas of the constructivist philosophy \cite{ben-ari_constructivism_1998}, such as LOGO \cite{solomon_history_2020} or Smalltalk \cite{kay_early_1993}.

Coming full circle to the origin of digital computers, exploitation of numerical analysis by undergraduate students performed with languages such as Python, that inherits the simplicity of LOGO and object orientation of Smalltalk, could provide them a useful calculation tool at almost any science or engineering course.
As digital pocket calculators freed students of repetitive arithmetic tasks, allowing to explore otherwise avoided problems, current programming languages provide symbolic and numerical analysis as well as plotting capabilities that allow to visualise and explore the mathematical solutions of any conceivable problem presented in university courses.