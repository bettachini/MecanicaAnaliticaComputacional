%% bare_jrnl.tex
%% V1.4b
%% 2015/08/26
%% by Michael Shell
%% see http://www.michaelshell.org/
%% for current contact information.
%%
%% This is a skeleton file demonstrating the use of IEEEtran.cls
%% (requires IEEEtran.cls version 1.8b or later) with an IEEE
%% journal paper.
%%
%% Support sites:
%% http://www.michaelshell.org/tex/ieeetran/
%% http://www.ctan.org/pkg/ieeetran
%% and
%% http://www.ieee.org/

\documentclass[journal]{IEEEtran}
\usepackage{amsmath,amsfonts}
\usepackage{algorithmic}
\usepackage{algorithm}
\usepackage{array}
\usepackage[caption=false,font=normalsize,labelfont=sf,textfont=sf]{subfig}
\usepackage{textcomp}
\usepackage{stfloats}
\usepackage{url}
\usepackage{verbatim}
\usepackage{graphicx}

%% escritura en castellano con todas los caracteres no ASCII
\usepackage[T1]{fontenc}  % 8-bit encoding. Acentuados como único caracter.
\usepackage[utf8]{inputenc}
\usepackage[spanish, es-nodecimaldot, es-tabla]{babel} % http://minisconlatex.blogspot.com.ar/2013/03/latex-en-espanol.html

%% biblatex
\usepackage[style = numeric, backend = biber, sorting = none, doi = false, isbn = false, url = true]{biblatex}
% \usepackage[defernumbers = true, style = numeric, backend = biber, sorting = none, doi = false, isbn = false, url = true]{biblatex}
% \usepackage[style = numeric, backend = biber, sorting = none]{biblatex}    % REFERENCIAS como section
\AtEveryBibitem{
    \clearfield{urlyear}
    \clearfield{urlmonth}
} % Do not show the "(visited on <date>)" on the references
\DefineBibliographyStrings{spanish}{}
\usepackage{csquotes}
\addbibresource{./edu.bib}
\renewcommand*{\bibfont}{\fontsize{9}{12}\selectfont}

\usepackage{xcolor}
\newcommand{\MR}[1]{{\color{magenta}#1}}



\begin{document}

\title{Experiences on a Computational Analitical Mechanics course based on code and inverted classroom}
%
%
% author names and IEEE memberships
% note positions of commas and nonbreaking spaces ( ~ ) LaTeX will not break
% a structure at a ~ so this keeps an author's name from being broken across
% two lines.
% use \thanks{} to gain access to the first footnote area
% a separate \thanks must be used for each paragraph as LaTeX2e's \thanks
% was not built to handle multiple paragraphs
%

\author{Víctor~A.~Bettachini,   Mariano~A.~Real,        and~Edgardo~Palazzo% <-this % stops a space
% \thanks{M. Shell was with the Department
% of Electrical and Computer Engineering, Georgia Institute of Technology, Atlanta,
% GA, 30332 USA e-mail: (see http://www.michaelshell.org/contact.html).}% <-this % stops a space
% \thanks{J. Doe and J. Doe are with Anonymous University.}% <-this % stops a space
% \thanks{Manuscript received April 19, 2005; revised August 26, 2015.}
}

% The paper headers
% \markboth{Journal of \LaTeX\ Class Files,~Vol.~14, No.~8, August~2015}%
% {Shell \MakeLowercase{\textit{et al.}}: Bare Demo of IEEEtran.cls for IEEE Journals}

% make the title area
\maketitle

\begin{abstract}
“Numerical Analysis” and “Programming Fundamentals'' are mentioned as “Knowledge Descriptors'' for a Mechanical Engineer in the “Proposal of standards for the second generation for engineering degrees accreditation in the Argentine Republic” 
%(“Propuesta de Estándares de Segunda Generación para la Acreditación de Carreras de Ingeniería en la República Argentina”) 
approved by the “Federal council of engineering deans” %(Consejo Federal de Decanos de Ingeniería, CONFEDI) 
in 2018, and best known as “Libro Rojo de CONFEDI”\cite{librorojo}. 
Regrettably after theory and use of these tools are learnt by students they are not fully exploited in courses at later years.

In this work the experience had at the subject Mecánica General (General Mechanics) of the UNLaM’s mechanical engineering degree third-year is described. Traditionally modeled systems are limited to those analytically solvable working on blackboard or paper. In this course the students solve their problem sets using Python language code, applying this century tools, such as library functions for symbolic and numerical analysis, plotting, etc.

All classes were conducted in full using Jupyter notebooks as a platform. In those notebooks code  is interbowen with graphical information and text including clear mathematical notation with LaTeX symbols. Students can re-use this same code to solve course’s problem sets as well as in future courses and their professional life.

The notebooks mentioned above run  on free software. Students operate on them with free to use web platforms that  allow concurrent commenting and editing among them and/or their professor.

The pandemic pushed us all to teach through a computer. After an initial adaptation period the students recognized the virtues of this methodology. Even assessments were more enriching than in a conventional course as it reached the complexity of simulating industrial-like mechanical systems.


% “Fundamentos de programación” y “Cálculo numérico” se mencionan entre los “descriptores de conocimiento” para un Ingeniero Mecánico en la “Propuesta de Estándares de Segunda Generación para la Acreditación de Carreras de Ingeniería en la República Argentina” aprobado por el Consejo Federal de Decanos de Ingeniería (CONFEDI) en 2018, mejor conocido como “Libro Rojo de CONFEDI”. Lamentablemente tras que la teoría y uso de estas herramientas son aprendidos por los alumnos no suelen aprovecharse en profundidad en cursos de años posteriores.

% En este trabajo se describe la experiencia que se tuvo en la asignatura Mecánica General del 3.er año de la carrera en la UNLaM. Tradicionalmente los sistemas modelados se limitan a los resolubles analíticamente por trabajar en pizarrón o papel. En este curso, los estudiantes resolvieron sus ejercicios utilizando código en lenguaje Python, haciendo uso de herramientas de este siglo, como bibliotecas de funciones para el cálculo simbólico, numérico, graficación, etc.

% Todas las clases se dictaron íntegramente usando cuadernos de Jupyter como plataforma. En estos se intercala código con información gráfica y texto incluyendo una clara notación matemática con simbología LaTeX. Este código es re-utilizable por el estudiante para resolver la ejercitación del curso con la misma herramienta, así como para ser aprovechado en asignaturas futuras y en su vida profesional.

% Estos cuadernos se ejecutan sobre software libre. Plataformas web de acceso gratuito a través del navegador permitieron a los estudiantes ejecutarlos en su hogar o trabajo, permitiendo comentar y editar en forma conjunta un mismo cuaderno entre alumnos y/o docentes.

% La pandemia nos forzó a enseñar a través de una computadora. Tras un periodo inicial de adaptación los estudiantes reconocieron las virtudes de esta metodología. Inclusive la evaluación fue más enriquecedora que en un curso convencional al alcanzar la complejidad de simular sistemas mecánicos similares a los industriales.\par\bigskip
The abstract goes here.
\end{abstract}

\begin{IEEEkeywords}
Mecánica, Código, Jupyter, Python.
\end{IEEEkeywords}

\section{Contexto}

Indicar área del conocimiento, temática, instituciones que coordinan el proyecto, empresas u organismos de financiamiento y algún otro dato de interés, de ser necesario. 

\section{Introduction}

The last three quarters, the SARS-CoV-2 pandemic forced the courses for Mechanical Engineering students at the National University of La Matanza (UNLaM) to be taught remotely. The fact that students were behind a computer during class was used to impose a pedagogical methodology that omitted traditional chalkboard, paper, and non-interactive computer presentations in favor of presenting theoretical concepts and practicing their use in activities on an interactive computer platform based on code written in the Python language.

The complexity of the code increases as new aspects affecting a mechanical system are contemplated from class to class. 
In a chalkboard and paper-based course, similar calculations are repeated in each new activity, while in a code-based course, it presents the advantage of its reuse. 
By making modifications to the code tested in previous classes with simple mechanical systems, the analysis capacity is expanded without the loss of time that would be required for writing from scratch the long set of calculations that an analysis in the Euler-Lagrange scheme of a more complex mechanical system requires.


In the post-pandemic return to the classroom the proposed methodology was further improved, now using an inverted classroom aproach \MR{incluir CITA}. The students are presented to online theory and examples to be read and physics problems to be started at home that must be finished during their time in the classroom with the help of teachers. In this way the theory and initial stages of the problem-solving are asynchronous, while they take advantage of the classroom time to ask issues regarding the problems and theory. 



%Source: Conversation with Bing, 11/23/2023
% % (1) How to Translate Languages in Python - Python Code. https://thepythoncode.com/article/translate-text-in-python.
% % (2) Text Translation with Google Translate API in Python - Stack Abuse. https://stackabuse.com/text-translation-with-google-translate-api-in-python/.
% % (3) Translate website to any specific language, on page load. https://stackoverflow.com/questions/13030153/translate-website-to-any-specific-language-on-page-load.
% % (4) undefined. http://domain.example.
% % (5) undefined. http://code.google.com/p/jquery-translate/.

% Los últimos tres cuatrimestres la pandemia de SARS-CoV-2 forzó a que los cursos para los estudiantes de Ingeniería Mecánica en la Universidad Nacional de La Matanza (UNLaM) se dicten en forma remota. El hecho de que los alumnos estén tras una computadora durante la clase se aprovechó para imponer una metodología pedagógica que obvió los tradicionales soportes pizarrón, papel y presentaciones informáticas no interactivas en favor de presentar conceptos teóricos y ejercitar su uso en actividades en una plataforma informática interactiva basada en código escrito en el lenguaje Python.
% La complejidad del código se incrementa a medida que se contemplan clase a clase nuevos aspectos que afectan a un sistema mecánico. En un curso basado en pizarrón y papel se repiten cálculos similares en cada nueva actividad mientras que en un curso basado en código este presenta la ventaja de su reutilización. Realizando modificaciones al código probado en clases anteriores con sistemas mecánicos simples se expande la capacidad de análisis sin la pérdida de tiempo que insumiría una escritura desde cero del largo conjunto de cálculos que insume un análisis en el esquema de Euler-Lagrange de un sistema mecánico más complejo.

%\begin{figure}[!t]
%\centering
%\includegraphics[width=2.5in]{myfigure}
% where an .eps filename suffix will be assumed under latex, 
% and a .pdf suffix will be assumed for pdflatex; or what has been declared
% via \DeclareGraphicsExtensions.
%\caption{Simulation results for the network.}
%\label{fig_sim}
%\end{figure}


%\begin{tabular}{|c||c|}
%\hline
%One & Two\\
%\hline
%Three & Four\\
%\hline
%\end{tabular}
%\end{table}
\section{About Computational Analytical Mechanics course in UNLaM}

En el plan de la carrera de grado en Ingeniería Mecánica del Departamento de Ingeniería e Investigaciones Tecnológicas (DIIT) de la UNLaM esta asignatura es el nexo entre las primeras propias de la especialidad con las básicas en las que se imparten herramientas de álgebra, análisis matemático, cálculo numérico, y mecánica Newtoniana. El esquema de correlatividades inmediatas a la asignatura Mecánica General, que muestra la figura \ref{fig:correlativas}, deja en claro que esta debe tener entre sus objetivos el mostrar al alumno cómo dichas herramientas tienen aplicación en su tema de interés.

\begin{figure}[!ht]
\centering
\includegraphics[width=3in]{figuras/correlativas.png}
\caption{Precedida de asignaturas de álgebra, análisis y física, Mecánica General es la primera en que se aplican tales conocimientos a la ingeniería mecánica.}
\label{fig:correlativas}
\end{figure}

La asignatura entrena a los alumnos en la habilidad de modelizar la física de sistemas mecánicos simples. Se entiende por modelizar el realizar una serie de procedimientos con los que se construye un esquema simplificado de la física partiendo de una evaluación semi-cuantitativa de las fuerzas y campos que actúan sobre el sistema así como de las ligaduras que limitan sus grados de libertad. Con tal información se priorizan algunas de estas y se descartan otras para arribar al esquema mencionado. Disponer de tal modelo permite:
\begin{itemize}
    \item elegir coordenadas generalizadas para describir los grados de libertad relevantes,
    \item escribir relaciones matemáticas entre estas que den cuenta de ligaduras,
    \item describir las fuerzas generalizadas que no sean efecto de campos (gravitatorio, electromagnéticos, etc.),
    \item y describir la energía potencial y cinética del sistema en su conjunto.
\end{itemize}

Tras realizar lo anterior se demuestra y se pone en práctica en el curso el formalismo de Euler-Lagrange para obtener un conjunto de ecuaciones diferenciales que describen la dinámica del sistema y/o los esfuerzos mecánicos que cada uno de sus componentes debe soportar en cada instante de tiempo.

De lo expuesto en los párrafos precedentes se evidencia que la temática del curso objeto de este trabajo está circunscrita a la convencional de los cursos de mecánica racional como se detalla en su literatura canónica de referencia \cite{landau}. No es en la temática sino en su metodología didáctica donde se hizo una innovación.

\subsection{El pizarrón, única herramienta didáctica}

Tradicionalmente los sistemas que se trabajan en los cursos de mecánica racional son relativamente simples para acotar el tiempo y/o dificultad de los cálculos de análisis matemático y/o de álgebra que requieren los pasos comentados en el párrafo anterior. Pero esta simplificación extrema lleva a un notorio salto en la complejidad de la que requiere modelar de dispositivos mecánicos, la dinámica de fluidos o a estructuras rígidas, las respectivas temáticas de las asignaturas subsiguientes a Mecánica General (ver figura \ref{fig:correlativas}).

La limitación a la complejidad la impone lo que el docente puede, en la duración de una clase, calcular en el pizarrón. Estos al irse borrando sucesivamente no sólo impide al alumno servirse de referencia de algo que ya no está a la vista sino que además le impone dedicar buena parte de su atención a no cometer errores al transcribir lo allí escrito en su cuaderno. Este soporte en papel a su vez limita la extensión y complejidad de los problemas que pueden proponerse al alumnado para ejercitar lo aprendido. 
Lo  que sintetiza el párrafo precedente no es otra cosa que el proceder en el dictado de clases de ciencias o ingeniería a nivel universitario que se reproduce casi en forma inalterada desde el siglo XIX hasta nuestros días.

\subsection{Herramientas didácticas informáticas}

Los pasos previos y posteriores al obtener un sistema de ecuaciones diferenciales que describen la dinámica y esfuerzos para un modelo mecánico complejo pueden realizarse con herramientas informáticas, pero que son distintas para cada caso.
Para resolver y analizar el resultado de las ecuaciones se aplican las herramientas aprendidas en la asignatura Cálculo Numérico, cursada previamente a Mecánica General, y las ubicuas de graficación para visualizar la evolución temporal de distintas magnitudes. Si bien es cierto que el cálculo numérico se aprovecha ocasionalmente en la ejercitación \cite{mirasso-raichman, caligaris-rodriguez}, este rara vez es utilizado por el docente durante la clase. Se pierde así una oportunidad de ejemplificar mejor y profundizar el análisis del comportamiento de los sistemas modelados.

Pero si es rara la aplicación del cálculo numérico durante las clases lo es aún más el uso de sistemas de álgebra computacional (Computer Algebra Systems o CAS, en inglés), que permiten automatizar todos los procedimientos matemáticos que requiere la modelización: desde definir grados de libertad, sistema de coordenadas, campos y fuerzas externas al modelo hasta construir el sistema de ecuaciones diferenciales para la dinámica. El utilizarlos para resolver cálculos de álgebra lineal y análisis matemático permite quitar el énfasis sobre estos y centrar la atención del docente y alumnos en la temática propia de la asignatura.

Pero la realidad cotidiana de nuestras aulas es que durante las clases dichos cálculos se continúan realizando manualmente en el pizarrón o en papel obviando el uso de la informática. Empeñarse en ese proceder en el nivel universitario sería análogo al de privar al alumnado del uso de calculadoras de bolsillo para realizar procedimientos aritméticos aprendidos en el nivel primario. Todo un anacronismo en la tercera década del siglo XXI.

\subsection{Reciclado del código}

Lo precedente puede interpretarse erróneamente como un mero llamado a utilizar la informática como un análogo de la calculadora de bolsillo, supliendo resultados de los cálculos que demanda la resolución de ejercicios en papel. Eso sería una infrautilización de tal recurso en la clase, repitiendo el patrón que sigue la mayor parte de los usuarios de computadoras que obvian en su uso cotidiano un aspecto fundamental de la informática.

La computadora digital se inventó en la Segunda Guerra Mundial con el fin de realizar cálculos numéricos que se definían manualmente en cada ejecución operando como una calculadora más rápida y con capacidad de automatizar algunos de los procesos de manipulación numérica.  Pero desde la mitad del siglo pasado adquirió la capacidad de operar bajo una serie de instrucciones almacenadas en su memoria sobre cómo procesar información tanto numérica como de otra índole. Tales instrucciones reciben el nombre de programa, y se escriben en un código que respeta la sintaxis de un determinado lenguaje.

Los lenguajes modernos de alto nivel permiten escribir un único código incorporando  todos los procedimientos que requiere la resolución y el análisis de un problema de mecánica racional. Esto abarca desde las aproximaciones asumidas para simplificar la física del mismo hasta el análisis con gráficas de su dinámica y esfuerzos mecánicos pasando por todos los cálculos algebraicos y numéricos intermedios.

Partiendo del objetivo de que los alumnos exploten esta herramienta el curso tuvo por metodología el obviar el trabajo en papel y en su lugar desarrollar la habilidad de escribir en un único código el conjunto de operaciones que requiere la resolución de ejercicios. En clase el docente explicó en forma sincrónica ejemplos de códigos que realizan todos los procedimientos requeridos para modelar un sistema mecánico. En cada clase se provee una guía de ejercicios resolubles haciendo pequeñas modificaciones al código provisto por el docente en esa fecha. Sucesivos ejercicios de complejidad creciente requieren pequeñas modificaciones respecto al código con que se resolvió el anterior. Esta reutilización del código permite un mejor aprovechamiento del tiempo y esfuerzo del alumno que en resoluciones en papel donde debe repetir procedimientos ya realizados en anteriores oportunidades. Clase a clase el alumnado construye una biblioteca de códigos con capacidades crecientes de análisis \cite{Barba2019}.

Hay que aclarar que no se forma al alumno en programación para que cree aplicaciones o algoritmos, lo que se denomina programming en inglés. Lo que se busca es que puedan codificar  (por coding en inglés) las instrucciones para que la computadora realice tareas específicas, en particular cálculos para la ingeniería mecánica.

Algunos alumnos archivan sus resoluciones de ejercicios resueltas en papel como referencia en caso de que se presente una problemática similar más adelante en el cursado de la carrera o en la vida profesional. En los hechos esto rara vez sucede y si recurren en el futuro a algún material relacionado a la asignatura es a su bibliografía, que por lo expuesto anteriormente carece de ejemplos adaptables a modelos más complejos que los comúnmente tratados en la asignatura. Por contrapartida un código es fácilmente aplicable al análisis de una problemática profesional análoga a las vistas en el curso. Al figurar en forma explícita las instrucciones para realizar cada paso del procedimiento es sencillo de revisar, expandir y modificar.

\section{Methods}

Desarrolle aquí los medios que se aplicarán para alcanzar los resultados durante el proyecto. De requerir subtítulos de menor nivel, utilizarlos siguiendo el estilo de la Norma IEEE.

\section{Tools used on this curse}

\subsection{Python, Sympy, Numpy, Scipy y Matplotlib}

El lenguaje de programación Python está por defecto desprovisto de capacidades de cálculo científico e ingenieril.  Esta es una decisión de diseño para hacer que tales funcionalidades deban ser agregadas por bibliotecas especializadas. El efecto de esta decisión es que el desarrollo de las mismas corre por cuenta de usuarios que las aplican en diversos ámbitos del desarrollo científico-tecnológico antes que por profesionales de las informática.

Las funciones del cálculo simbólico las provee la biblioteca Sympy. Se aprovecha en particular su módulo Mechanics que facilita la generación de ecuaciones para la dinámica de sistemas de cuerpos rígidos con múltiples grados de libertad y en variados sistemas de referencia \cite{simpy}.

Los sistemas de ecuaciones diferenciales se resuelven por métodos numéricos apoyados en las funciones para la manipulación de elementos algebraicos de la biblioteca Numpy \cite{numpy} y de los algoritmos de optimización e integración numérica de Scipy \cite{SciPy}.

El análisis en ingeniería de resultados numéricos son usualmente interpretados con representaciones gráficas. Esta capacidad la proveen las funciones de la biblioteca Matplotilb \cite{matplotlib}.

\subsection{Cuadernos de Jupyter}

El entorno usado en el curso para ejecutar código es la aplicación basada en la web del Proyecto Jupyter llamada JupyterLab cuyo  formato de documento es el cuaderno (notebook) Jupyter \cite{Kluyver2016jupyter}. Este alterna secciones independientes denominadas celdas. Las de entrada son de código (en variados lenguajes, Python es solo uno de los posibles) o de  anotaciones, como muestra la figura \ref{fig:jupyter}. Esta última variante de celdas se escriben en el lenguaje de  marcado Markdown \cite{markdown} que permite incrustar: texto y/o expresiones matemáticas en formato \LaTeX intercaladas, y contenido multimedia: enlaces web, imágenes, reproductores de video o sonido.

\begin{figure}[!ht]
\centering
\includegraphics[width=3.5in]{figuras/screenshot_JupyterLab.png}
\caption{Un cuaderno de Jupyter es un conjunto de celdas. Estas son en formato Markdown o de código ejecutable. Las primeras pueden contener texto,  expresiones  matemáticas o contenido multimedia. Las segundas líneas de código en variados lenguajes de programación. Intercalando títulos en las celdas Markdown se genera el índice (a la izquierda) que facilita la ubicación dentro del documento.}
\label{fig:jupyter}
\end{figure}

La utilización de sintaxis \LaTeX para la simbología matemática provee una notación clara estandarizada bajo los lineamientos de la American Mathematical Society \cite{ams}.
El resultado de la ejecución de una celda de código muestra al usuario el resultado que el mismo instruye a la computadora imprimir. En el curso estas últimas incluyen tanto los comandos para realizar cálculos así como la resolución de un sistema de ecuaciones no lineales que se imprime en la última celda del cuaderno mostrado en la figura 2.  

\subsection{Ejecución de Jupyter en línea}

No se impone a los estudiantes el instalar ningún software para cursar la materia en su dispositivo informático. Solo requieren utilizar un navegador web estándar para utilizar alguno de los servicios que ejecutan cuadernos de Jupyter en línea. Este puede tratarse de una instalación de JupyterHub del Proyecto Jupyter en servidores propios de la universidad o en nubes comerciales, o en su defecto de alguno de los servicios que ofrecen alternativas incluso gratuitas como, entre otras, CoCalc, IBM Watson o Google Colaboratory. De estas se ha utilizado esta última en las últimas ediciones del curso tras cerrar Microsoft su servicio gratuito Azure Notebooks.

El servicio Google Colaboratory, coloquialmente sólo Colab,  presenta como conveniencia el poder ejecutar cuadernos alojados en un repositorio Git gerenciado por el servicio en línea GitHub. Basta una modificación en el URL de un cuaderno para que este apunte a un navegador web a ejecutarle en Colab \cite{colab}. El trabajo con cuadernos Jupyter en este servicio puede realizarse en forma concurrente por parte de varios alumnos y/o docentes. También pueden incluirse comentarios cuya actualización es reportada por correo electrónico lo que es útil para la corrección de los ejercicios pues pueden indicarse la ubicación de errores en el código como muestra la figura \ref{fig:colab}.

\begin{figure}[!ht]
\centering
\includegraphics[width=3.5in]{figuras/comentariosColab.png}
\caption{El sitio web Google Colaboratoy permite editar y ejecutar cuadernos Jupyter en forma concurrente entre alumnos y docentes además de incluir comentarios. Esta última característica es útil para las correcciones.}
\label{fig:colab}
\end{figure}

\subsection{Repositorio Git}

El mencionado repositorio en GitHub está organizado en sendas carpetas por clase del curso, como muestra la figura \ref{fig:github}. Cada una de estas aloja el correspondiente material teórico y ejercicios en el formato de cuadernos Jupyter además de  guías de ejercicios y algún apunte ocasional en el formato de documento portátil conocido por su sigla en inglés PDF. Este ordenamiento facilita tanto al docente como a los alumnos una vista de conjunto del material de cada temática así como el verificar las eventuales actualizaciones del mismo. De esta forma el material del curso es de acceso público haciéndolo disponible para ser utilizado a interesados \cite{repositorio-victor} mientras cumplan con citar su origen y no darle uso comercial como indica su licencia Creative Commons CC-BY-NC-SA bajo el que está publicado \cite{creative}.

\begin{figure}[!ht]
\centering
\includegraphics[width=3.5in]{figuras/repositorioGithub.png}
\caption{Los alumnos encuentran el material ordenado en sendos directorios por clase.}
\label{fig:github}
\end{figure}

\subsection{Sistema de gestión de aprendizaje}

En la UNLaM se utiliza la plataforma de comunicaciones de negocios Microsoft Teams para suplir la interacción en el aula con los alumnos con videoconferencias. Luego de terminada cada clase el video de las mismas se guarda en el almacenamiento en línea Microsoft OneDrive. Enlaces a estos y a los materiales de la clase alojados en el repositorio Git se  compartimentan en lo que el sistema llama canales respetando la misma numeración y denominación que en el repositorio Git. La figura \ref{fig:teams} muestra los contenidos que encabezan los desplegados para la décima clase.

En cada canal se incluyen enlaces a:
\begin{itemize}
    \item guía de ejercicios prácticos
    \item algún eventual apunte en PDF
    \item ambas vías para ver los cuadernos Jupyter, la interactiva en Colab o estática en nbviewer
    \item invitación a la videoconferencia o a su video una vez esta terminó
\end{itemize}

Microsoft Teams provee también los rudimentos de un sistema de gestión de aprendizaje (LMS por sus siglas en inglés) al permitir asignar tareas a alumnos con fechas límites de aceptación por parte del sistema. Los alumnos pueden cargar al sistema un enlace a su cuaderno en Colab o el mismo en formato .ipynb en el caso de que no se permita modificación del mismo con posterioridad a una fecha por cuestiones de evaluación.

\begin{figure}[!ht]
\centering
\includegraphics[width=3.5in]{figuras/notasTeams.png}
\caption{Sendos canales por clase presentan los enlaces a su material.}
\label{fig:teams}
\end{figure}

\section{Cruse chronology}

Of the 16 weekly online meetings throughout the semester, 13 of them present new topics. Some of these topics were selected to illustrate the progression of the course.

\textbf{Class 1.} Review of Newtonian mechanics, analysis, and algebra required to obtain the ideal pendulum dynamics revisiting the approximations and calculations seen in Physics 1. This theory material is distributed in this and subsequent classes in a Jupyter notebook. Students are encouraged to review the LaTeX notation with which the teacher writes mathematical formulas such as those shown in Figure \ref{fig:clase1pendulo}.

\begin{figure}[!ht]
\centering
\includegraphics[width=3.5in]{figuras/clase1péndulo.png}
\caption{The theory is presented in Jupyter notebooks. All mathematical formulas are expressed in standardized \LaTeX notation that students can edit or copy for their own purposes.}
\label{fig:clase1pendulo}
\end{figure}

In addition to the reiteration of what has already been seen in previous courses, in this first class, we already advance in the use of code to analyze results. Figure \ref{fig:clase1graficos} shows instructions for the Matplotlib library to graph the solution for the ideal pendulum dynamics.


\textbf{Class 3.} Starting from this class, the Sympy library is applied in class for automatic symbolic calculation. Figure \ref{fig:clase3sympy} shows how to calculate the kinetic energy of a system with two generalized coordinates differentiated in its reference system.

\begin{figure}[!ht]
\centering
\includegraphics[width=3.5in]{figuras/clase3sympy.png}
\caption{First symbolic calculations using the SymPy library.}
\label{fig:clase3sympy}
\end{figure}

\textbf{Class 4.} The Euler-Lagrange equations allow students to obtain the equations that describe the dynamics of a system. Figure \ref{fig:clase4euler} shows how functions of the Sympy library facilitate obtaining such equations for a system with two degrees of freedom.

\begin{figure}[!ht]
\centering
\includegraphics[width=3.5in]{figuras/clase5EulerLagrange.png}
\caption{Application of SymPy functions to generate the Euler-Lagrange differential equations that describe the dynamics of a system.}
\label{fig:clase4euler}
\end{figure}

So far, code has been used to perform the same steps that are solved on a blackboard or paper in a conventional rational mechanics course to arrive at differential equations that are only solved for trivial cases. In contrast, using Sympy quickly solves complex systems for accelerations as a function of generalized coordinates and velocities as shown in Figure \ref{fig:clase4ac}. Performing such a task manually would require a non-negligible amount of time and effort even for this system with only two degrees of freedom.

\begin{figure}[!ht]
\centering
\includegraphics[width=3.5in]{figuras/clase4Aceleraciones.png}
\caption{The resolution of systems of differential equations of certain complexity is avoided in conventional courses. In this course, it only takes a couple of lines of code with functions of the Sympy library.}
\label{fig:clase4ac}
\end{figure}

\textbf{Class 5.} Students passed a numerical calculus course to enroll in this course where such knowledge will be used. In class, the fundamentals of numerical resolution methods for differential equations are reviewed and how they would be implemented in a state vector notation suitable for efficient processing. Such a review is presented to students with the same methodology as for other topics, in Jupyter notebooks that students can edit, as shown in Figure \ref{fig:clase5res}.

\begin{figure}[!ht]
\centering
\includegraphics[width=3.5in]{figuras/clase5Euler.png}
\caption{Prior to proceeding with numerical resolution of differential equations, a review of its fundamentals is presented in Jupyter notebooks.}
\label{fig:clase5res}
\end{figure}

Immediately after the review of fundamentals, the functions of the scientific calculation library Scipy are shown in action to efficiently obtain solutions for the dynamics of a two-degree-of-freedom system as illustrated in Figure \ref{fig:clase5sol}.

\begin{figure}[!ht]
\centering
\includegraphics[width=3.5in]{figuras/clase5Soluciones.png}
\caption{The system of equations for the dynamics of a two-degree-of-freedom system is numerically solved with functions of the SciPy library.}
\label{fig:clase5sol}
\end{figure}


The generalized positions and velocities obtained numerically in the range of times of interest are graphically represented. Figure \ref{fig:clase5rep} shows such a representation that serves to discuss with the students whether the behavior of the system is consistent with what can be predicted from a qualitative analysis of this simple system. Confirming that the symbolic and numerical calculation tools used obtain correct results gives confidence in them in view of applying them to more complex systems.

\subsection{Class 7.} Non-conservative forces are incorporated into the codes, which ultimately are the majority of those that can affect an industrial mechanical device. As a first example, the analogy of pendulum oscillations is extended to a damped system shown in Figure \ref{fig:clase7esquema} to analyze non-conservative forces. In this system, a linear damper acts with velocity. This non-conservative force could not be analyzed with the code of previous classes.}
\label{fig:clase7esquema}
\end{figure}


The figure \ref{fig:clase7amo} shows the graph that allows analyzing the dynamics calculated with the same procedure and code that has been used in the previous classes.

\begin{figure}[!ht]
\centering
\includegraphics[width=3.5in]{figuras/clase7amortiguado.png}
\caption{The range of analyzable factors is gradually extended. In this system, a linear damper with velocity acts. This non-conservative force could not be analyzed with the code of previous classes.}
\label{fig:clase7amo}
\end{figure}

\textbf{Next classes.} The usual syllabus of a course in rational mechanics is completed by focusing on extensive systems analyzed within the framework of the rigid body and the analysis of forced oscillations in systems with multiple degrees of freedom. Towards the end of the course, students have already developed the ability to autonomously analyze "realistic" systems in terms of being more similar to mechanical devices existing in the industry. To capture this, they are proposed to calculate the torques that the motors of a highly simplified industrial robotic arm must perform so that it performs a sequence of movements. Examples of the result of students' work in response to this proposal are shown in figure \ref{fig:robotarm}.

\begin{figure}[!ht]
\centering
\includegraphics[width=3.5in]{figuras/robotArm.png}
\caption{For an industrial mechanical arm to perform even a simple movement, its motors must apply a sequence of torques. Students calculate them in a work that reflects their mastery of analytical and computer tools whose use was learned in the course.}
\label{fig:robotarm}
\end{figure}

Source: Conversation with Bing, 11/27/2023
(1) . https://bing.com/search?q=%22translate+from+spanish+to+english%3a+%22%2buser_input.
(2) How to Translate Languages in Python - Python Code. https://thepythoncode.com/article/translate-text-in-python.
(3) undefined. http://www.bing.com/translator/?ref=TThis&text=&from=es&to=en.
(4) undefined. http://domain.example.








% De los 16 encuentros semanales en línea a lo largo del cuatrimestre en 13 de ellos se presentan nuevos temas. De estos se seleccionaron algunos para ilustrar la progresión del curso.

% \textbf{Clase 1.} Repaso de la mecánica Newtoniana, el análisis y el álgebra requeridos para obtener la dinámica del péndulo ideal revisitando las aproximaciones y cálculos vistos en Física 1. Este material de teoría se distribuye tanto en esta como en las subsiguientes clases en un cuaderno Jupyter. Se anima a los alumnos a revisar la notación LaTeX con las que el docente escribe fórmulas matemáticas como las que se muestran en la figura \ref{fig:clase1pendulo}. 

% \begin{figure}[!ht]
% \centering
% \includegraphics[width=3.5in]{figuras/clase1péndulo.png}
% \caption{La teoría se presenta en cuadernos Jupyter. Todos las fórmulas matemáticas se expresan en notación estandarizada \LaTeX que los alumnos pueden editar o copiar para sus propios fines.}
% \label{fig:clase1pendulo}
% \end{figure}

% En adición a la reiteración de lo ya visto en cursos anteriores en esta primera clase ya se avanza en el uso de código para analizar resultados. La figura \ref{fig:clase1graficos} muestra instrucciones para que la biblioteca Matplotlib grafique la solución para la dinámica del péndulo ideal.

% \begin{figure}[!ht]
% \centering
% \includegraphics[width=3.5in]{figuras/clase1péndulo.png}
% \caption{Desde la primera clase se hace explícito a los estudiantes el código utilizado para el análisis de sistemas. Aquí las funciones de Matplotib para graficar la dinámica de un péndulo ideal.}
% \label{fig:clase1graficos}
% \end{figure}

% \textbf{Clase 3.} A partir de esta clase se aplica en clase la biblioteca Sympy para el cálculo simbólico automático. La figura \ref{fig:clase3sympy} muestra cómo para calcular la energía cinética de un sistema con dos coordenadas generalizadas se diferencia en su sistema de referencia.

% \begin{figure}[!ht]
% \centering
% \includegraphics[width=3.5in]{figuras/clase3sympy.png}
% \caption{Primeros cálculos simbólicos utilizando la biblioteca SymPy.}
% \label{fig:clase3sympy}
% \end{figure}

% \textbf{Clase 4.} Las ecuaciones de Euler-Lagrange permiten a los alumnos obtener las ecuaciones que describen la dinámica de un sistema. La figura \ref{fig:clase4euler} muestra como funciones de la biblioteca Sympy facilitan el obtener tales ecuaciones para un sistema de dos grados de libertad.

% \begin{figure}[!ht]
% \centering
% \includegraphics[width=3.5in]{figuras/clase5EulerLagrange.png}
% \caption{Aplicación de funciones de SymPy para general las ecuaciones diferenciales de Euler-Lagrange que describen la dinámica de un sistema.}
% \label{fig:clase4euler}
% \end{figure}

% Hasta aquí se ha utilizado el código para realizar los mismos pasos que en un curso de mecánica racional convencional se resuelven en pizarrón o papel para arribar a ecuaciones diferenciales que solo se resuelven para casos triviales. En contrapartida utilizando Sympy se resuelven rápidamente sistemas complejos para aceleraciones en función de coordenadas y velocidades generalizadas como se muestra en la figura \ref{fig:clase4ac}. Realizar tal tarea manualmente insumiría un tiempo y esfuerzo no despreciable inclusive para este sistema con meros dos grados de libertad.

% \begin{figure}[!ht]
% \centering
% \includegraphics[width=3.5in]{figuras/clase4Aceleraciones.png}
% \caption{La resolución de sistemas de ecuaciones diferenciales de cierta complejidad se evita en cursos convencionales. En este curso solo insume un par de líneas de código con funciones de la biblioteca Sympy.}
% \label{fig:clase4ac}
% \end{figure}

% \textbf{Clase 5.} Los estudiantes aprobaron un curso de cálculo numérico para poder inscribirse a este curso en el que se hará uso de tales conocimientos. En clase se repasan los fundamentos de los métodos de resolución numérica de ecuaciones diferenciales y cómo se implementarían en una notación de vectores de estado adecuada para un eficiente procesamiento. Tal repaso se presenta a los estudiantes con la misma metodología que para los otros temas, en cuadernos Jupyter que los alumnos pueden editar, como se muestra en la figura \ref{fig:clase5res}.  

% \begin{figure}[!ht]
% \centering
% \includegraphics[width=3.5in]{figuras/clase5Euler.png}
% \caption{Previo a proceder a la resolución numérica de ecuaciones diferenciales se presenta, en cuadernos de Jupyter, un repaso de sus fundamentos.}
% \label{fig:clase5res}
% \end{figure}

% Inmediatamente tras el repaso de fundamentos se muestran en acción las funciones de la biblioteca de cálculo científico Scipy para obtener eficientemente las soluciones para la dinámica de un sistema de dos grados de libertad como se ilustra en la figura \ref{fig:clase5sol}.

% \begin{figure}[!ht]
% \centering
% \includegraphics[width=3.5in]{figuras/clase5Soluciones.png}
% \caption{Se resuelve numéricamente el sistema de ecuaciones para la dinámica de un sistema de dos grados de libertad con funciones de la biblioteca SciPy.}
% \label{fig:clase5sol}
% \end{figure}

% Las posiciones y velocidades generalizadas en el rango de tiempos de interés obtenidas numéricamente se representan gráficamente. La figura \ref{fig:clase5rep} muestra tal representación que sirve para discutir con los alumnos si el comportamiento del sistema se condice con el que puede predecirse de un análisis cualitativo de este sistema simple. El comprobar que las herramientas utilizadas de cálculo simbólico y numérico obtienen resultados correctos confiere confianza en los mismos en vistas de aplicarles a sistemas más complejos.

% \begin{figure}[!ht]
% \centering
% \includegraphics[width=3.5in]{figuras/clase5Representación.png}
% \caption{Visualización de resultados obtenidos por cálculo numérico. Corroborando que lo representado corresponde con un análisis cualitativo de la dinámica del sistema se crea confianza en los alumnos en esta herramienta.}
% \label{fig:clase5rep}
% \end{figure}

% \subsection{Clase 7.} Se incorpora a los códigos el análisis de fuerzas no conservativas que a fin de cuentas son la mayoría de las que pueden afectar a un dispositivo mecánico industrial. Como primer ejemplo se extiende la analogía de las oscilaciones del péndulo a un sistema amortiguado que se muestra en la figura \ref{fig:clase7esquema}.

% \begin{figure}[!ht]
% \centering
% \includegraphics[width=3.5in]{figuras/clase7esquema.png}
% \caption{Paulatinamente se extiende el rango de factores analizables. En este sistema actúa un amortiguador lineal con la velocidad. Esta fuerza no conservativa no podía analizarse con el código de clases precedentes.}
% \label{fig:clase7esquema}
% \end{figure}

% La figura \ref{fig:clase7amo} muestra la gráfica que permite analizar la dinámica calculada con el mismo procedimiento y código que se viene utilizando en las clases precedentes.

% \begin{figure}[!ht]
% \centering
% \includegraphics[width=3.5in]{figuras/clase7amortiguado.png}
% \caption{Paulatinamente se extiende el rango de factores analizables. En este sistema actúa un amortiguador lineal con la velocidad. Esta fuerza no conservativa no podía analizarse con el código de clases precedentes.}
% \label{fig:clase7amo}
% \end{figure}

% \textbf{Siguientes clases.} El temario habitual de un curso de mecánica racional se va completando centrándose en sistemas extensos analizados en el marco del cuerpo rígido y el análisis de oscilaciones forzadas en sistemas de múltiples grados de libertad. Hacia finales del curso los alumnos ya han desarrollado la habilidad de analizar en forma autónoma sistemas “realistas” en términos de ser más semejantes a dispositivos mecánicos existentes en la industria. Para plasmar esto último se les propone que calculen los torques que deben realizar los motores de un muy simplificado brazo robótico industrial para que este realice una secuencia de movimientos. Ejemplos del resultado del trabajo de alumnos en respuesta a esta propuesta se muestran en la figura \ref{fig:robotarm}.

% \begin{figure}[!ht]
% \centering
% \includegraphics[width=3.5in]{figuras/robotArm.png}
% \caption{Para que un brazo mecánica industrial  realice aún un simple movimiento se requiere que sus motores apliquen una secuencia de torques. Los alumnos realizan el cálculo de los mismos en un trabajo que plasma su dominio de las herramientas analíticas e informáticas cuyo uso fue aprendido en el curso.}
% \label{fig:robotarm}
% \end{figure}

\section{Results and objectives}

Coloque aquí los resultados alcanzados y los objetivos en curso o futuros del proyecto.

\section{Discusssion}

\input{discusion}

\section{End discussion}

La exposición de teoría y la ejercitación práctica tomaron distintas ventajas de la metodología de este curso.

Clases de teoría:
\begin{itemize}
    \item La exposición en pizarrón o en una presentación en que los alumnos están concentrados en transcribir lo allí escrito se reemplazó por código que pueden reutilizar para resolver sus ejercicios.
    \item En las clases en línea cada palabra del docente durante la clase queda registrada en video  liberando al alumno de la toma de notas.
    \item El docente puede cambiar el código durante la clase para corregir un error o graficar otro aspecto de la temática.
\end{itemize}

Ejercicios de práctica:
\begin{itemize}
    \item En papel una variación sobre un ejercicio resuelto anteriormente obliga a reiterar tediosos cálculos similares. Con código basta con modificar ligeramente el mismo para atender al nuevo caso.
    \item En forma remota varios alumnos pueden trabajar concurrentemente en la resolución en un mismo ejercicio.
    \item Los alumnos pueden alertar al docente a toda hora vía el LMS de un inconveniente que enfrenten en la resolución de un ejercicio. El docente puede dedicarle tiempo y detenimiento en el momento que encuentre propicio a diferencia del acotado tiempo de consultas del que se dispone en el aula.
    \item Los docentes pueden comentar y corregir el mismo código sobre el que está trabajando el alumno inclusive en tiempo real.
\end{itemize}


\section{Conclussions}

% This is a summary of the main drivers of the course:
This course differs from conventional ones in two ways:
\begin{itemize}
    \item Code-centric% Advantages of Code-Based Learning:
    \begin{itemize}
        \item Avoids the repetitive nature of blackboard or paper based calculations. 
        \item By iteratively modifying previously tested code (initially designed for simpler mechanical systems), students expand their analytical capabilities.
        \item The complexity of the code evolves alongside the mechanical system’s intricacies introduced each class.
        \item This approach eliminates the need to \textit{start from scratch} when dealing with the extensive calculations required for analyzing complex mechanical systems using the Euler-Lagrange formalism.
        \item All systems used are currently available online on a non-cost basis, from the student point of view. Being based on free software, if any of them is later placed behind a paywall, it would be simple to run them from on the premise servers.
    \end{itemize}
    \item Flipped classroom
    \begin{itemize}
        \item Students are provided with online theory and example problems to study before weekly meetings. These asynchronous activities save classroom time for discussions and problem solving.
        \item During synchronic meetings they can rise to teachers any questions related to theory or problem-solving so they can finish their exercise sets.
        \item All exercises are turned-in for evaluation. Compliance is tracked with an online learning management system. 
    \end{itemize}
\end{itemize}
Currently, there is limited statistical data available on the impact of the course and the described methodologies.
However, feedback from students consistently indicates a high level of satisfaction, especially with the code-driven aspect of the course.
Additionally, students express interest in the final examination as it provides an opportunity to apply both their presentation skills and the knowledge acquired throughout the course.

In relation to the flipped classroom model, students acknowledge that it requires more effort, but a majority of them agree that it is a positive and beneficial implementation.
as former students assesed that the tools employed in the course were usuful to them in subsequent subjects and professional lives, the authors had consequently gained confidence on having choosed the current approach over traditional ones.


\section{Akgnowledgements}

Los autores quieren agradecer a la coordinación de la carrera de Ingeniería Mecánica del DIIT-UNLaM qué ha acompañado el desarrollo e implementación de este curso.



% ============================================
% if have a single appendix:
%\appendix[Proof of the Zonklar Equations]
% or
%\appendix  % for no appendix heading
% do not use \section anymore after \appendix, only \section*
% is possibly needed

% use appendices with more than one appendix
% then use \section to start each appendix
% you must declare a \section before using any
% \subsection or using \label (\appendices by itself
% starts a section numbered zero.)
%
\appendices
\section{Proof of the First Zonklar Equation}
Appendix one text goes here.
% you can choose not to have a title for an appendix
% if you want by leaving the argument blank
\section{}
Appendix two text goes here.
% use section* for acknowledgment
\section*{Acknowledgment}
The authors would like to thank...

\printbibliography[title= Referencias, heading=bibintoc]

% \begin{thebibliography}{1}

% \bibitem{IEEEhowto:kopka}
% H.~Kopka and P.~W. Daly, \emph{A Guide to \LaTeX}, 3rd~ed.\hskip 1em plus
%   0.5em minus 0.4em\relax Harlow, England: Addison-Wesley, 1999.

% \end{thebibliography}

\end{document}


