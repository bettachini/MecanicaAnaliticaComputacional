In this paper, we present different technichs applied to a course on Computational Analytical Mechanics for Mechanical Engineering students. The course is designed to teach students Analythical Mechanics and how to take advantage of programming languages as a tool to solve problems in physics. The course is produced using online free tools such as Python, Markdown and LaTeX by using Jpyther notebooks in Google Colab. The course is hosted on Github and Microsoft Teams, which provide an excellent way to manage and arrange the information given and also keep track of the work required to the students. 
Students are not required to have prior knowledge of programming languages, but they are expected to modify the code given by the professor to solve the syllabus problems. The course was initially taught completely online since started during the SARS-CoV-2 pandemic, but now it is given in the classroom. During the return to the classroom it was further improved by appling an inverted classroom technique, where students are given all the material for the class beforehand. They must read the theory and start working on problems to then use the synchronous time with the teachers to ask questions on the theory and finishing solving the problems. We find that reducing the ammount of homework but seting all problems as  mandatory, greatly improved the students response. The course is finished by solving a final complex problem, where the students solve forces and torques on a mechanical robotic arm. They are required to defend it in a short presentation to teachers, improving their presentation and sythesis skills.

