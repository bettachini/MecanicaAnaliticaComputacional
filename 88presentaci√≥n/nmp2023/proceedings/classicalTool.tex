The course employs the most traditional principles of rational mechanics, as outlined in standard reference literature \cite{landau}.
Modeling is understood as a series of procedures with which a simplified scheme of physics is constructed based on a semi-quantitative evaluation of the forces and fields that act on the system as well as the constraints that limit its degrees of freedom.
With such information, some of these are prioritized and others are discarded to arrive at the aforementioned scheme.
Having such a model allows:
\begin{itemize}
    \item to choose generalized coordinates to describe the relevant degrees of freedom,
    \item to write mathematical relationships between them that account for constraints,
    \item to describe generalized forces that are not the effect of fields (gravitational, electromagnetic, etc.),
    \item and to describe the potential and kinetic energy of the system as a whole.
\end{itemize}

After performing the above, the Euler-Lagrange formalism is demonstrated and put into practice in the course to obtain a set of differential equations that describe the dynamics of the system and/or the mechanical stresses that each of its components must withstand over time.