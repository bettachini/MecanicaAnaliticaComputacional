The psychologist J.C.R. Licklider stated in the 1950s that 85\% of his ``thinking'' time was in fact used in information related tasks such as finding it or plotting graphs; an observation that set him to become an advocate of interactive computing and push the creation of ARPANET the antecessor to today's INTERNET \cite{waldrop_dream_2001}.
In the 21st century, it is not unusual to find university courses following a routine where professors transcribe lessons by heart to blackboards or repeat slide presentations. Students then transcribe this information once again into their paper or digital notebooks, a methodology largely unaltered since its introduction in the 19th century.
The situation is aggravated in science and engineering courses, where the same calculations and drawings are made repeatedly, not only in the classroom but also when exercises based on the subject of the class require performing very similar tasks to those done by the teachers.

This excercise on repeated transcription can be seen as a waste time due to the ubiquitous availability of technologies product of the information revolution during the last half of the 20th century propelled by the efforts of Licklider and colleagues.
Trying to avoid this our university course harness the power of interactive computing for in-class lessons as well as for solving problem sets using the same digital platform. 
All material that the teaching staff want to share to students is in a digital format that not only contains all theory on the subject but code that solves problems related to it.
A document containing executable code for each lesson is stored in a public repository from where the students generate their own copy.
The copy's code is fully modifiable, and students are required to do so in order to solve the problem sets proposed by the teaching staff in a procedure akin to recycling.
The matematician S. Papert, a pioneer of employing computers in a constructivist framework for education, onec stated that learning takes place when the learner takes charge of the operation \cite{papert_childrens_1993}.
Following this advise, the problem-sets in our course are built to lead the student gradually to becoming autonomous by reusing not the code provided by the teaching staff but his own code to address increasingly more complex problems.

In the current century, the fact that university courses do not ubiquitously employ coding to save time an effort of everyone in our classrooms is a situation alike the rejection of pocket calculators in 1970s argumenting that abilities on arithmetics would be jeopardised \cite{roberts_impact_1980}.
Nowadays after students learnt arithmetics at elementary school they can freely use pocket calculators.
The same should be the case after they passed their calculus and algebra courses, they should be able to employ freely their computational counterparts in all their following courses.
Not only would the focus of their effort be diverted from those automatable calculations, but they could also tackle with the tools of numerical analysis problems beyond what can be solved on a blackboard or on paper.
