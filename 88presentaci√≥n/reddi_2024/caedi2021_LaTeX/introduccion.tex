Los últimos tres cuatrimestres la pandemia de SARS-CoV-2 forzó a que los cursos para los estudiantes de Ingeniería Mecánica en la Universidad Nacional de La Matanza (UNLaM) se dicten en forma remota. El hecho de que los alumnos estén tras una computadora durante la clase se aprovechó para imponer una metodología pedagógica que obvió los tradicionales soportes pizarrón, papel y presentaciones informáticas no interactivas en favor de presentar conceptos teóricos y ejercitar su uso en actividades en una plataforma informática interactiva basada en código escrito en el lenguaje Python.
La complejidad del código se incrementa a medida que se contemplan clase a clase nuevos aspectos que afectan a un sistema mecánico. En un curso basado en pizarrón y papel se repiten cálculos similares en cada nueva actividad mientras que en un curso basado en código este presenta la ventaja de su reutilización. Realizando modificaciones al código probado en clases anteriores con sistemas mecánicos simples se expande la capacidad de análisis sin la pérdida de tiempo que insumiría una escritura desde cero del largo conjunto de cálculos que insume un análisis en el esquema de Euler-Lagrange de un sistema mecánico más complejo.