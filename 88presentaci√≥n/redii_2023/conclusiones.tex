El uso de herramientas informáticas durante la clase para la resolución de ejercicios de mecánica racional  aliviaría, en especial al alumnado, de tediosos cálculos en pizarrón o papel que distraen del temario propio de la asignatura. El no hacerlo se justificaba por cuestiones de disponibilidad de equipamiento, en particular en las universidades del tercer mundo. La pandemia de SARS-CoV-2 forzó a realizar cursos a través de internet que demostraron donde tales recursos estaban disponibles: en los hogares y lugares de trabajo de alumnos y docentes.

El curso objeto de este trabajo instrumentó una metodología para presentar los conceptos de teoría y su puesta en práctica en ejercicios a través de la escritura de código que capitaliza conocimientos de los alumnos adquiridos en materias previamente cursadas: “Fundamentos de programación” y “Cálculo numérico”. Los términos entre paréntesis son “descriptores de conocimiento” para un Ingeniero Mecánico en el “Libro Rojo de CONFEDI” \cite{librorojo}. No hacer uso de los mismos en materias posteriores a aquellas en las que se aprendieron es un desperdicio del esfuerzo de sus docentes y alumnos. Docentes en universidades reconocidas a nivel mundial también lo han entendido así al iniciar en esta última década cursos de ingeniería con idénticas metodología y herramientas que las presentadas en este trabajo \cite{barba_teaching_2019}  [4,17,18].

 Es el convencimiento de los autores que la metodología de este curso presenta una mayor utilidad al estudiante en vistas a próximas asignaturas y su vida profesional respecto a la que pueden brindar clases presenciales. Las ventajas citadas en este trabajo de un curso basado en código por sobre una cursada presencial les lleva a recomendar la continuidad de esta metodología independientemente de si las condiciones sanitarias vuelven a permitir cursos con la metodología convencional basados en tecnología del siglo XIX.