\item % frecuencias audibles de perturbación en gas
\begin{minipage}[t][2.5cm]{0.57\textwidth}
Una explosión causa una rápida perturbación en la presión atmosférica.
Tras un rápido incremento sigue una rápida caída exponencial.
El incremento suele modelizarse como una función lineal.
Para simplificar este ejercicio el decaimiento no lo consideraremos exponencial sino lineal y simétrico con el incremento:
\end{minipage}
\begin{minipage}[c][1cm][t]{0.3\textwidth}
	\centering
	\input{\detokenize{figuras/perturbación}}
\end{minipage}
\begin{enumerate}[label=\alph*)]
  \item Calcule el espectro de tal modelo en el dominio de la frecuencia \(\hat{\delta p}(\nu) \).
	Ayuda: \(2 \sin^2{\left(\frac{x}{2} \right) } = 1 - \cos{(x)}\)
	\item De una medición se estimó la duración de la perturbación en forma aproximada \(2 \tau \simeq \SI{1.3d-4}{s}\).
	En tal caso se observa que la potencia \(\propto |\hat{\delta p}(\nu)|^2\):
	\begin{itemize}
		\item Cerca del límite superior de la audición humana (\(\SI{20}{\kilo\hertz}\)) es despreciable frente a las de \(\nu\) menores.
		\item En \SI{5}{\kilo\hertz} sería aproximadamente la mitad de la pico en \(\nu=0\).
		\item En \SI{10}{\kilo\hertz} es menos de un octavo de la pico en \(\nu=0\).
	\end{itemize}
	Para refinar el conocimiento sobre \(\tau\) se propone que:
	\begin{enumerate}
		\item \textbf{Calcule} la expresión de \(|\hat{\delta p}(\nu)|^2\) y \textbf{grafíquela}.
		Está bien hacerlo en forma cualitativa siempre que se respete la dependencia que tiene la expresión y se  verifiquen las observaciones anteriores.
		\item Un análisis espectral muestra que la potencia en \SI{5}{\kilo\hertz} \textbf{es exactamente la mitad de la de pico}, lo que permite refinar la estimación de \(\tau\).\\
		Escriba la relación (no la resuelva) que permite obtener \(\tau\).
		Para calcular un valor para el pico no se compliquen calculando \(\lim_{\nu \to 0} |\hat{\delta p}(\nu)|^2\), estímenlo asumiendo
		\begin{itemize}
			\item \(2 \tau \simeq \SI{1.3d-4}{s}\)
			\item un \(\delta p_\text{máx}\) arbitrario, e.g. \(1\), y obviamente el mismo en \(|\hat{\delta p}(\nu = \SI{5}{\kilo\hertz})|^2\)
			\item que \(|\hat{\delta p}(\nu = 0)|^2 \simeq |\hat{\delta p}(\nu = \SI{1d-6}{\hertz})|^2\) 
		\end{itemize}
	\end{enumerate}
\end{enumerate}

